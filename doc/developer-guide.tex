\documentclass[a4paper,12pt]{article}
\usepackage{listings}
\usepackage{color}
\usepackage{textcomp}
\usepackage{hyperref}
\usepackage{underscore}
\definecolor{listinggray}{gray}{0.9}
\definecolor{lbcolor}{rgb}{0.9,0.9,0.9}
\lstset{
	backgroundcolor=\color{lbcolor},
	tabsize=4,
	rulecolor=,
	language=c++,
        basicstyle=\tiny,
        upquote=true,
        aboveskip={1.5\baselineskip},
        columns=fixed,
        showstringspaces=false,
        extendedchars=true,
        breaklines=true,
        prebreak = \raisebox{0ex}[0ex][0ex]{\ensuremath{\hookleftarrow}},
        frame=single,
        showtabs=false,
        showspaces=false,
        showstringspaces=false,
        identifierstyle=\ttfamily,
        keywordstyle=\color[rgb]{0,0,1},
        commentstyle=\color[rgb]{0.133,0.545,0.133},
        stringstyle=\color[rgb]{0.627,0.126,0.941},
}
\renewcommand{\familydefault}{\sfdefault}
\hypersetup{
  colorlinks,
  citecolor=black,
  filecolor=black,
  linkcolor=black,
  urlcolor=black
}
\author{by the Matahari Team}
\title{Matahari Developer's Guide}
\begin{document}
\maketitle
\newpage
\tableofcontents
\newpage
\abstract{This document is intended for developers who wish to 
implement an enterprise messaging framework within 
their organization. The main focus of this article is
the fundamentals behind Matahari \& QMF.}
\section{Getting Started}
\subsection{Requirements}
\subsubsection{Packages}
Packages to install for source based compilation and rpm building
\begin{lstlisting}[language=bash]
projects/> yum groupinstall "Fedora Packager" "Development Libraries" "Development Tools" "MinGW cross-compiler"
\end{lstlisting}
Optional packages may include: mingw32-cxxtest

\subsubsection{Checkout Source Code}
\begin{lstlisting}[language=bash]
projects/> git clone git://github.com/matahari/matahari.git
\end{lstlisting}
\subsubsection{Develop against Fedora packages}
\begin{lstlisting}[language=bash]
projects/> yum install matahari* mingw32-matahari*
\end{lstlisting}
\subsection{Verify Matahari Builds}
It's always a good idea to build and test the existing 
code to make sure no problems crop up before attempting 
to develop an agent. There are 2 commands provided for 
cross compilation.
\subsubsection{Building for Linux platform}
\begin{lstlisting}[language=bash]
$ make linux.build
\end{lstlisting}
\subsubsection{Building for Windows platform}
\begin{lstlisting}[language=bash]
$ make windows.build
\end{lstlisting}

Once building is complete our unittest's should run 
automatically on the Linux platform.\footnote{Windows builds will need their unittests run on a Windows host.}
Before beginning the journey to create an agent make sure that these core unit test \textbf{never fail}.
\subsection{Verify our coding guidelines}
See \url{https://github.com/matahari/matahari/wiki/Coding-Guidelines}
\newpage
\section{Preparing Agent}
\subsection{The End Result}
What is it exactly the agent in question is going to be responsible for?
Once that is defined an API needs to be outlined. For simplicities sake 
the API will only consist of generating a report when an agent is disconnected 
from the broker. A simulation of a crash can be considered when an agent 
unintentially disconnects.

\subsection{Creating the API}
All public API routines are kept within \textbf{matahari/src/include/matahari/} and 
\textbf{crashreporter.h} will be the header file used for defining our API.

\subsubsection{Create the routines that will be publicly accessible.}

\lstinputlisting[caption=crashreporter.h,language=C]{../src/include/matahari/crashreporter.h}

\textbf{mh\_report\_crash} is the routine that will be accessed by the broker when
the agent disconnected.\footnote{All public API routines are prefixed with mh\_}

\subsection{Schema Definition}
A schema describes the structure of management data. Each agent provides a 
schema that describes its management model including the object classes, 
methods, events, etc. that it provides. In the current QMF distribution, the 
agent's schema is codified in an XML document. In the near future, there will 
also be ways to programatically create QMF schemata.\footnote{Taken from 
https://cwiki.apache.org/qpid/qpid-management-framework.html\#QpidManagementFramework-Schema}\\

CrashReporter schema will be located alongside our agent code in the folder
\textbf{matahari/src/crashreporter/}\footnote{This is where the main agent code
resides}\\

Create the \textbf{schema.xml} file with the following contents
\lstinputlisting[caption=schema.xml,language=XML]{../src/crashreporter/schema.xml}

\subsection{Build Instructions with cmake}
The build tool of choice is CMake. Setting up our agent to be included into the distribution
should contain a \textbf{CMakeLists.txt} which gives the build process the necessary
instructions.

\lstinputlisting[caption=CMakeLists.txt,language=make]{../src/crashreporter/CMakeLists.txt}
\newpage
\section{Creating the Agent}
\subsection{Agent initialization code}
\lstinputlisting[caption=crashreporter-qmf.cpp,language=c++]{../src/crashreporter/crashreporter-qmf.cpp}
\subsection{Writing Test Cases}
\subsection{Testing the Agent}
\subsubsection{API level test with CxxTest}
\subsubsection{Functional testing with Beaker}
\newpage
\section{Finalizing the Agent}
\subsection{Build}
\subsection{Using Mock}
\subsection{Source Building}
\subsection{Publish}
Those wishing to publish their agents for the general public can follow the
below link which describes necessary steps in becoming a Fedora contributor 
and submitting the package for inclusion into the next release.

Visit\\
\url{https://fedoraproject.org/wiki/Join\_the\_package\_collection\_maintainers#How\_to\_join\_the\_Fedora\_Package\_Collection\_Maintainers.3F}
\end{document}
